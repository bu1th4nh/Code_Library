% /*==========================================================================================*\
% **                        _           _ _   _     _  _         _                            **
% **                       | |__  _   _/ | |_| |__ | || |  _ __ | |__                         **
% **                       | '_ \| | | | | __| '_ \| || |_| '_ \| '_ \                        **
% **                       | |_) | |_| | | |_| | | |__   _| | | | | | |                       **
% **                       |_.__/ \__,_|_|\__|_| |_|  |_| |_| |_|_| |_|                       **
% \*==========================================================================================*/
% 
%   LaTeX HUST Project Template v2.05 by bu1th4nh. Copyright (c) 2022 bu1th4nh. All right reserved
%   Author: @bu1th4nh (Bùi Tiến Thành)
%   Date: 2022/05/05 15:47
%   URL: https://github.com/bu1th4nh
% 
\documentclass[12pt,a4paper]{report}
\usepackage[utf8]{vietnam} % Sử dụng tiếng việt
\usepackage[top=3.5cm, bottom=3cm, left=3.5cm, right=2cm] {geometry} % Canh lề trang
\usepackage{graphicx} % Cho phép chèn hỉnh ảnh
\usepackage{fancybox} % Tạo khung box
\usepackage{indentfirst} % Thụt đầu dòng ở dòng đầu tiên trong đoạn
\usepackage{amsthm} % Cho phép thêm các môi trường định nghĩa
\usepackage{latexsym} % Các kí hiệu toán học
\usepackage{amsmath} % Hỗ trợ một số biểu thức toán học
\usepackage{amssymb} % Bổ sung thêm kí hiệu về toán học
\usepackage{amsbsy} % Hỗ trợ các kí hiệu in đậm
\usepackage{array} % Tạo bảng array
\usepackage{enumitem} % Cho phép thay đổi kí hiệu của list
\usepackage{subfiles} % Chèn các file nhỏ, giúp chia các chapter ra nhiều file hơn
\usepackage{titlesec} % Giúp chỉnh sửa các tiêu đề, đề mục như chương, phần,..
\usepackage{chngcntr} % Dùng để thiết lập lại cách đánh số caption,..
\usepackage{pdflscape} % Đưa các bảng có kích thước đặt theo chiều ngang giấy
\usepackage{afterpage}
\usepackage{capt-of} % Cho phép sử dụng caption lớn đối với landscape page
\usepackage{multirow} % Merge cells
\usepackage{fancyhdr} % Cho phép tùy biến header và footer
\usepackage{setspace}
\usepackage{parskip}
\usepackage{booktabs}
\usepackage[boxruled, 
lined, 
algochapter, 
resetcount, 
procnumbered, 
linesnumbered]{algorithm2e} % Thêm mã giả

% \usepackage{minted}
\usepackage{float}


\usepackage[nomath]{lmodern} % Chuyển font code cho Latex
\usepackage{mathptmx} % Thay font cho math mode của Latex
\newcolumntype{L}{>{\centering\arraybackslash}m{2.3cm}}
\numberwithin{equation}{section}

%=======================================================
\usepackage[pdftex, % Sử dụng PDF TeX
bookmarks=true, % Tạo bookmarks trong tập tin PDF
colorlinks=false, % Chữ có màu
pdfencoding=auto, % Tự động điều chỉnh encoding của PDF
unicode=true, % Sử dụng Unicode
pdffitwindow=true, % Fit cho vừa cửa sổ
pdfstartview={FitW}, % Zoom file PDF cho vừa khít với nội dung
pdftoolbar=false, % Ẩn đi tool bar trong PDF viewer
pdfmenubar=false % Ẩn đi menu bar trong PDF viewer
]{hyperref}

\usepackage[all]{hypcap} % Cho phép tham chiếu chính xác đến hình ảnh và bảng biểu
\counterwithin{figure}{chapter} % Đánh số hình ảnh kèm theo chapter. Ví dụ: Hình 1.1, 1.2,..

% Phụ lục 
\usepackage[toc,page]{appendix} % Thêm phụ lục
\renewcommand{\appendixname}{Phụ lục}
\renewcommand{\appendixtocname}{Phụ lục}
\renewcommand{\appendixpagename}{Phụ lục}

% Tài liệu tham khảo
\usepackage[sorting=none,maxnames=99,giveninits=true,doi=true,url=true]{biblatex}
\addbibresource{report-bib.bib}
\DefineBibliographyStrings{english}{%
bibliography = {Tài liệu tham khảo},
references = {Tài liệu tham khảo},
}

% Dãn cách dòng & đánh số phần
\onehalfspacing
\setlength{\parskip}{6pt}
\setlength{\parindent}{15pt}
\renewcommand{\baselinestretch}{1.5}
\renewcommand{\thesection}{\arabic{section}}



% Đầu trang và chân trang
\pagestyle{fancy}
\fancyhf{}
\lhead{\footnotesize\nouppercase\leftmark} % Bên trái header: tên chapter
\cfoot{\footnotesize\thepage} % Giữa footer: số trang
\renewcommand{\headrulewidth}{2pt}
\renewcommand{\footrulewidth}{2pt}


% Định dạng section/subsection
\newcommand{\titlesize}{\fontsize{18pt}{23pt}\selectfont}
\newcommand{\subtitlesize}{\fontsize{16pt}{21pt}\selectfont}
\titleclass{\part}{top}
\titleformat{\part}[display]
  {\normalfont\huge\bfseries}{\centering}{20pt}{\Huge\centering}
\titlespacing{\part}{0pt}{2em}{1em}
\titlespacing{\section}{0pt}{\parskip}{0.5\parskip}
\titlespacing{\subsection}{0pt}{\parskip}{0.5\parskip}
\titlespacing{\subsubsection}{0pt}{\parskip}{0.5\parskip}


\setcounter{secnumdepth}{3} % Cho phép subsubsection trong report
\setcounter{tocdepth}{3} % Chèn subsubsection vào bảng mục lục

% Hình vẽ
\graphicspath{{./rpt-img/}}
\usepackage{svg} % Chèn ảnh vector
\usepackage{tikz} % cái này để vẽ bảng cho đẹp
\usetikzlibrary{external}

% Các gói phụ trợ cho hình vẽ
\usepackage{mathdots} 
\usepackage{yhmath}
\usepackage{cancel}
\usepackage{color}
\usepackage{siunitx}

% Cài đặt TikZ
\usetikzlibrary{fadings}
\usetikzlibrary{patterns}
\usetikzlibrary{shadows.blur}
\usetikzlibrary{shapes}

% Các dịnh nghĩa, định lý, hệ quả
\newtheorem{theorem}{Định lý}[chapter]
\newtheorem{result}[theorem]{Kết quả}
\newtheorem{corollary}{Hệ quả}[theorem]
\newtheorem{lemma}[theorem]{Bổ đề}
\newtheorem{num_algo}[theorem]{Thuật toán}

\renewcommand\qedsymbol{$\blacksquare$} % Cài đặt QED



%=======================================================
\begin{document}

    \newgeometry{top=2cm, bottom=2cm, left=3cm, right=2cm}
    \fontsize{14pt}{12pt}\selectfont 
\begin{titlepage}
\centering
\begin{minipage}{0.8\textwidth}
	\begin{tikzpicture}[remember picture,overlay,inner sep=0,outer sep=0]
     \draw[blue!70!black,line width=4pt] ([xshift=-1.5cm,yshift=-2cm]current page.north east) coordinate (A)--([xshift=1.5cm,yshift=-2cm]current page.north west) coordinate(B)--([xshift=1.5cm,yshift=2cm]current page.south west) coordinate (C)--([xshift=-1.5cm,yshift=2cm]current page.south east) coordinate(D)--cycle;
\draw ([yshift=0.5cm,xshift=-0.5cm]A)-- ([yshift=0.5cm,xshift=0.5cm]B)--
     ([yshift=-0.5cm,xshift=0.5cm]B) --([yshift=-0.5cm,xshift=-0.5cm]B)--([yshift=0.5cm,xshift=-0.5cm]C)--([yshift=0.5cm,xshift=0.5cm]C)--([yshift=-0.5cm,xshift=0.5cm]C)-- ([yshift=-0.5cm,xshift=-0.5cm]D)--([yshift=0.5cm,xshift=-0.5cm]D)--([yshift=0.5cm,xshift=0.5cm]D)--([yshift=-0.5cm,xshift=0.5cm]A)--([yshift=-0.5cm,xshift=-0.5cm]A)--([yshift=0.5cm,xshift=-0.5cm]A);


     \draw ([yshift=-0.3cm,xshift=0.3cm]A)-- ([yshift=-0.3cm,xshift=-0.3cm]B)--
     ([yshift=0.3cm,xshift=-0.3cm]B) --([yshift=0.3cm,xshift=0.3cm]B)--([yshift=-0.3cm,xshift=0.3cm]C)--([yshift=-0.3cm,xshift=-0.3cm]C)--([yshift=0.3cm,xshift=-0.3cm]C)-- ([yshift=0.3cm,xshift=0.3cm]D)--([yshift=-0.3cm,xshift=0.3cm]D)--([yshift=-0.3cm,xshift=-0.3cm]D)--([yshift=0.3cm,xshift=-0.3cm]A)--([yshift=0.3cm,xshift=0.3cm]A)--([yshift=-0.3cm,xshift=0.3cm]A);

   \end{tikzpicture}
\begin{center}
    \vspace{10pt}
    
    \textbf{TRƯỜNG ĐẠI HỌC BÁCH KHOA HÀ NỘI}
    
    \vspace{7pt}
    \textbf{VIỆN TOÁN ỨNG DỤNG VÀ TIN HỌC}
\end{center}
\vspace{30pt}
\begin{center}
    \includegraphics[scale=0.2]{rpt-img/LOGO_HUST.png}
    
    \vspace{40pt}
    \fontsize{18pt}{17pt}\selectfont 
    \textbf{BÁO CÁO MÔN HỌC} 
    
    \vspace{7pt}
    \textbf{XXXX XXXX XXXX XXXX}
\end{center}
\begin{flushleft}
    \fontsize{14pt}{18pt}\selectfont  
    \textbf{\textsl{ĐỀ TÀI:}}
\end{flushleft}
\begin{center}
    \fontsize{25pt}{25pt}\selectfont 
    \textcolor{magenta}{\sffamily \textbf{\bf\Large Tên đề tài}}\\
    % {\bf\Large Bài toán phân biệt và bài toán phân loại}\\
    % {\bf\large (Discrimination and Classification Problem)}\\
\end{center}

\vspace{1cm}
\textbf{Giảng viên hướng dẫn: }

\vspace{1cm}
\textbf{Nhóm sinh viên thực hiện: Nhóm 5 }
\vspace{0.5cm}
\begin{center}
\begin{tabular}{p{0.5\textwidth}c}
\begin{bfseries}Bùi Tiến Thành\end{bfseries}&  \begin{bfseries}20190081\end{bfseries}\\
\begin{bfseries}A\end{bfseries}& \begin{bfseries}xxxxxxxx\end{bfseries}\\
\begin{bfseries}B\end{bfseries}& \begin{bfseries}xxxxxxxx\end{bfseries}\\
\end{tabular}
\end{center}
\vspace{2.5cm}
\begin{center}
    \textbf{Hà Nội, tháng xx năm xxxx}
\end{center}
\end{minipage}
\end{titlepage}


    % \input{rpt-chapters/test.tex}
    \restoregeometry

    \pagestyle{empty}
    \input{rpt-chapters/acks}
    \newpage
    \tableofcontents % Mục lục
    \addcontentsline{toc}{chapter}{Mục lục}
    \newpage
    \listoffigures % Danh sách các hình ảnh
    \addcontentsline{toc}{chapter}{Danh sách các hình ảnh}
    \newpage
    \listoftables % Danh sách bảng biểu
    \addcontentsline{toc}{chapter}{Danh sách bảng biểu}
    \input{rpt-chapters/intro}
    \newpage
    \pagestyle{fancy}

    \input{rpt-chapters/chapter1}
    \input{rpt-chapters/chapter2}
    \input{rpt-chapters/chapter3}
    
    \newpage
    \pagestyle{plain}
    \nocite{*} % Tùy chọn cho phép in tất cả danh mục tài liệu tham khảo, kể cả tài liệu không được tham chiếu
    \printbibliography
    \addcontentsline{toc}{chapter}{Danh mục tài liệu tham khảo}
    
    \newpage
\appendix
\addappheadtotoc
\renewcommand{\thesection}{\Alph{section}}
\section{Các chương trình được sử dụng}
    \par Các chương trình trong báo cáo và slide được lưu tại các liên kết này: \url{} 

\section{Mã nguồn báo cáo và bài trình chiếu}
    \par Mã nguồn báo cáo và slide được lưu tại đây: \url{}

\section{Hướng dẫn sử dụng chương trình}

   

\end{document}
% Created by bu1th4nh
% Powered, inspired and motivated by EDM, Counter-Strike:Global Offensive and Disney Princesses